\section{Theory}

	Acoustic waves in \textbf{closed} liquids cause periodic high and low density regions in them. The spacing between high-density and low-density regions of ultrasonic waves in the MHz range is similar to the spacing in diffraction gratings. Since these density changes in liquids will cause changes in the index of refraction of the liquid, it can be shown that parallel light passed through the exciting liquid will be diffracted much as if it had passed through a grating. The experiment can be used as an indirect method to measure the velocity of sound in various liquids. The phenomenon of interaction between light and sound waves in a liquid is called the \textbf{DebyeSears effect}.

	The successive separations between two compressions or rarefactions are equal to the wavelength of the ultrasonic wave, $\lambda_u$ in the liquid. Due to reflections at the sides of the tank or the container, a stationary wave pattern is obtained with nodes and antinodes at regular intervals. In these periodic high and low density regions, the liquid acts as a diffraction grating.

	If $\lambda_u$ denotes the wavelength of sound in the liquid, $\lambda$ the wavelength of incident light in air and $\theta_n$ is the angle of diffraction of $n^{th}$ order, then we have:

	% \vspace{-4mm}
	$$d\sin\theta_n = n\lambda$$

	Here, the liquid acts as the diffraction grating. So, here, the $d (= (e+b) = \frac{1}{N})$ becomes equal to $\lambda_u$. Thus:

	% \vspace{-4mm}
	$$\lambda_u\sin\theta_n = n\lambda$$

	Now, if $\nu$ is the frequency of the crystal, the velocity $V_u$ of the ultrasonic wave in the liquid will be:

	% \vspace{-4mm}
	$$V_u = \nu\lambda_u$$