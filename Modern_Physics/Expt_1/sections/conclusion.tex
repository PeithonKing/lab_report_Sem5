\section{Discussions}

	\begin{itemize}
		\item The value for electron charge has a large error in the dynamic method (135\%). The main reason for this is, the lowest value of ne calculated didn't have n=1, it must have been some multiple. Assuming n=2 for the least value will give us $e = 1.88\times10^{-19}C$. This result has an error of $17.5\%$. Which is quite small an error.
		\item Hence through the experiment we have proved the quantization of charge.
		\item The large error in the dynamic method is because it requires measurement of more quantities- the rise and fall time, so the errors due to both are factored in the total error, compared to the balancing method where only one measurement is to be made.
		\item Several droplets were observed changing trajectories or moving in all axes rather than in a straight line, which could not be accounted for in the measurements and must have contributed to the error. While the buoyant forces and weight are always in one direction for good approximations, the viscous force and electric force will depend on the horizontal levelling of the plates with the cylinder. As a result, the levelling should be corrected ahead of time to produce electric force in the same direction as mg.
		\item The droplet should not be lost sight of during measurements. The droplet may be observed as reducing in size or disappearing due to movement along other axes and can be controlled by refocusing on it with the telescope.
		
	\end{itemize}
