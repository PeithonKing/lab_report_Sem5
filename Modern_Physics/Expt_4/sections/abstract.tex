\section{Abstract} 
By the quantum description of energy states, the emission spectra of metals are due to the excitation and de-excitation of electrons through the various discrete energy levels of their bound states. As the electrons de-excite, they emit light of energy corresponding to the energy gap. Since a spectrum is unique to an element, it is used for its detection. By using a constant deviation prism, which deviates the dispersed light beams at a minimum deviation of $90 \degree$, we can observe and study the different wavelengths of light emitted by the source. In the experiment, we will be studying the emission spectra of metals (Brass and Copper) and the absorption spectrum of iodine vapours.
