\section{Results \& Conclusions}
	Average Excitation energy $(\Delta U = 18.75 pm 3.31 eV)$

	% \subsection{Sources of Error}
	\begin{itemize}
		\item The observed value 18.8 eV is much closer to the 3p excited states range (18.4-19.0 eV) than the 3s states range (16.6-16.9 eV), thereby proving that the probability of transition from ground state to 3-p state is more than the 3-s states. There is a 0.26\% error between the experiment value 18.75 and average 18.7 eV energy of the 3p states.
		\item However presence of values such as 16.5 eV shows that the excitation to the 3s states cannot be ignored completely. And the actual transitions are a combination of the two types of transition.
		\item The luminous zones correlate with the minimas of the curve as it is there almost whole of the energy is utilized in exciting the electrons.
		\item The negative values of current occurs when $U_{AG}$ is greater than the anode voltage and the incoming electrons do not sufficient leftover kinetic energy to pass through the braking voltage, thereby leading to a reverse current.
	\end{itemize}
