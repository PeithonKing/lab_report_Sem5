\section{Calculations}

	From the observation tables we can see that the grating number is $$15000/in = \frac{15000}{0.0254}/m = 590551/m$$.
	
	\vspace{-1cm}
	\subsection{Wavelength for 1st order fringe:}
		Average angle $(\theta_1) = (\frac{20.291+20.292+20.315+20.319}{4})^\circ = 20.304^\circ$
	
		\vspace{3mm}
		So, $\lambda_{1} = \frac{\sin\theta_1}{mN} = \frac{\sin20.304}{1 \times 590551} = 5.876 \times 10^-7 = 587.6 \; nm$

	\subsection{Wavelength for 2nd order fringe:}
		Average angle $(\theta_1) = (\frac{44.281+44.276+44.318+44.305}{4})^\circ = 44.295^\circ$
	
		\vspace{3mm}
		So, $\lambda_{1} = \frac{\sin\theta_1}{mN} = \frac{\sin44.295}{2 \times 590551} = 5.912 \times 10^-7 = 591.2 \; nm$

	\subsection{Difference in the two wavelengths}
		$$\Delta\lambda = \frac{591.2-587.6}{2} \; nm = 3.6 \; nm$$

	\subsection{Angular Dispersive Power of the grating:}
		$\dfrac{d \theta}{d \lambda} \bigg|_{m=1} = \frac{1 \times 15000}{\cos20.304} = 15.999 \times 10^3 rad / in$

		\vspace{5mm}
		$\dfrac{d \theta}{d \lambda} \bigg|_{m=2} = \frac{1 \times 15000}{\cos44.295} = 20.957 \times 10^3 rad / in$
