\section{Theory}

	\subsection{Sodium Spectrum}
	The sodium spectrum consists mainly of two wavelengths \textbf{589.0 nm} and \textbf{589.6 nm}. Using an appropriate diffraction grating the wavelength of these two lines can be determined.

	\subsection{Diffraction Grating}
	A grating is an arrangement with many small slits of the same width separated by equal opaque spaces known as diffraction gratings. For N parallel slits, each with width $e$ and separated by an opaque space of width $b$, the diffraction pattern consists of diffraction modulated interference fringes. The quantity $(e+b)$ is called the grating element and $N \; (= \frac{1}{e+b})$ is the number of slits per unit length, which could typically be 300 to 12000 lines per inch. For a large number of slits, the diffraction pattern consists of extremely sharp (practically narrow lines) principal maxima, together with weak secondary maxima in between the principal maxima. The various principal maxima are called orders.

	For polychromatic incident light falling normally on a plane transmission grating, the principal maxima for each spectral wavelength are given by:

	$$(e+b) \sin\theta = \pm m \lambda$$
	\begin{equation}
		\therefore \lambda = \abs{\frac{\sin\theta}{m \times N}}
	\end{equation}\label{eqn:1}

	Where m is the order of principal maximum and $\theta$ is the angle of diffraction. Angular dispersive power:

	The angular dispersive power of the grating is defined as the rate of change of angle of diffraction with the change in wavelength. It is obtained by differentiating Eqn. 1 and is given by

	\begin{equation}
		\frac{d\theta}{d\lambda} = \frac{m}{(e+b)\cos\theta} = \frac{m\times N}{\cos\theta}
	\end{equation}\label{eqn:2}