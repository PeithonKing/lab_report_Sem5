\section{Abstract} 
	A Fabry-Perot interferometer was used in the experiment as an etalon to measure the separation between the plates and as a stereoscopic tool to determine the diode laser's wavelength. It makes use of the phenomenon of multiple beam interference to create circular fringes with an equal inclination because the path differences of the interfering beams are the same for beams with an equal inclination. They have an interference pattern that is noticeably sharper than a Michelson interferometer due to multiple reflections and shared optical paths.
