\section{Discussion}

\begin{itemize}
	\item Compared to the literature value of 632.8 nm, the experimental value of wavelength 656.76 nm lands and error of $\approx 3.7\%$.
	\item The different errors include backlash error in the travelling microscope while taking measurements for the width of slits, which in the experiment was minimal and can be minised by tightening the screws. Also the microscope has to be levelled so that the slit edges are not tilted with respect to the marker on the lens.
	\item The other mistake includes parallax when seeing the pattern and writing their position on the graph since observing the pattern directly without interrupting it is not feasible. Also, the image may be unfocussed. The diffraction pattern may also not be perfectly horizontal, in which case the screen must be adjusted or the distance determined trigonometrically.
	\item Laser should not be viewed directly with the naked eye. The laser should be used with proper safety goggles.
	\item The pattern will be challenging to obtain if the laser is substituted with a monochromatic sodium lamp source. Because each filament of the sodium bulb acts as a separate source, the beam would be incoherent. Even though only a small piece of the wavefront passes through the slit, temporal coherence is established. Since light is monochromatic, spatial coherence will exist. However, the intensity may not be sufficient enough to generate a pattern at such a great distance. As a result, the coherence time and length are minimal, and obtaining the pattern will be tricky.
	\item We utilised the position of minimas for the calculation section since the formula simplifies to a simpler form in the case of minimas, but the form of the equation in the case of maximas involves tan theta, which is more difficult to calculate. Furthermore, the human eye is better able to distinguish between low and high intensity areas. As a result, the dark portions are more visible, whilst the brightest intensity zone is not. The larger bandwidth of the maximas prevents them from being used for other calculations.
	\item \textbf{Missing Order:} In the double slit experiment, when the diffraction pattern becomes zero, even if an interference bright fringe should have been observed in that region, its intensity remains zero. This phenomenon is called missing order.

\end{itemize}