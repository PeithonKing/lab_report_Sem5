\section{Abstract}
	When the slit width is on the order of the wavelength of light and the separation between the source, slit, and screen is practically infinite, Franhouffer diffraction occurs. By doing so, it is possible to determine the wavelength of the light used, the corresponding slit width, and the wire thickness by examining the minimas close to the central maxima. The phenomenon occurs when a single slit is replaced by a double slit, grating, or both. By differentiating the diffraction and interference minimas, we can determine the width of the slits and opaque region. Through the use of laser light passing through single, double, and thin wire in this experiment, we investigated the Fraunhofer diffraction phenomenon and discovered the aforementioned quantities.