\section{Result and Discussion}

    $$\mu_B = (3.998 + 0.83) \times 10^{-24} Am^2$$

    \begin{itemize}
        \item The value of $\mu_B$ obtained is quite away from the actual value. I think the primary reason is, these magnets are not properly calibrated with the data graph provided to us. Maybe these magnets are old and have degraded over time. To support my claim, I calculated the values of  the magnetic fields which should have been to obtain an accurate value of $\mu_B$.
        \begin{table}[H]
            \centering
            \resizebox{\columnwidth}{!}{%
            \begin{tabular}{|c|c|c|}
                \hline
                \begin{tabular}[c]{@{}c@{}}pole separation\\ (mm)\end{tabular} & \begin{tabular}[c]{@{}c@{}}B (from graph)\\ (mT)\end{tabular} & \begin{tabular}[c]{@{}c@{}}B fro accurate\\ $\mu_B$  (mT)\end{tabular} \\ \hline
                40 & 1900 & 1232.086 \\ \hline
                42 & 1700 & 985.1424 \\ \hline
                44 & 1050 & 791.7417 \\ \hline
                45 & 800  & 702.3776 \\ \hline
            \end{tabular}%
            }
        \end{table}

        We should have been given equipments to calibrate the magnetic fields with distances before the experiment. This would have given us a more accurate value of $\mu_B$.

        \item According to me, the change in magnetic field should be done with a change in current and an electomagnet and not a distance from permanent magnet. This would have given us a more accurate value of $\mu_B$.
    \end{itemize}