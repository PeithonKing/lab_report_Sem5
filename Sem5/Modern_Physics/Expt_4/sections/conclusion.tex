\section{Conclusions}
\begin{itemize}
\item For the emission spectra of metals, comparing the literature value of the wavelengths with the observed wavelength, we find that they are quite close. For Copper, we found all the major wavelengths mentioned in the literature. Brass is a mixture of copper and zinc. We observed both the characteristic wavelengths of copper and zinc in brass. This tells us, that, in Brass, the chemical properties of the two metals remain protected.

\item For the absorption spectra of Iodine, the value of $f$ is close to the literature value of $41\; N m^{-1}$ while $D_0$ is far from the literature value of $1.54\;eV/molecule$. The error in bond dissociation energy can be associated with the fact that during the experiment the lines green weren't visible and the intensity continued to decrease, thereby lowering the range of observation and thus the difference of energy between the highest and lowest energy state.

\item The large value of error in the calculation of $f$ is due to the large value of least count of the device. Also, there must have been a lot of random errors involved. The Source of one such error might be due to the fact that the fringes were not perfectly focussed, and while taking a reading, we might have taken reading at a position slightly beside the position of the actual minima.
\end{itemize}