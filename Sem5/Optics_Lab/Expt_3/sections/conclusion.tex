\section{Conclusion}
	We see that our values for the 1st order fringes vary a lot from the literature values mentioned in the ~\cite{manual}. On the other hand, it is quite close to the literature values for the 2nd order fringes. This is because the 1st order fringes are much closer to the center of the diffraction pattern, and thus the error in the measurement of the angle is much more. The 2nd order fringes are much farther from the center, and thus the error in the measurement of the angle is relatively low. This is also the reason why the 1st order fringes are much more sensitive to the error in the measurement of the angle.

	The 1st order fringes being close to the central maxima, are more affected by \textbf{backlash error} in the device. The 2nd order fringes being farther from the central maxima, are less affected by backlash error in the device. Hence those values are more accurate.

	In terms of observation, I think we can improve the experiment by using digital meters for measuring the angles in spectrometer, that shall decrease much of the random error.

	We should use devices with lesser values of least count, so that the error in the measurement of the angle is less. To do this, we can use spectrometers with larger diameters. We used a spectrometer with a larger diameter for \href{https://github.com/PeithonKing/lab_report_Sem5/blob/main/Optics%20Lab/Expt_2/main.pdf}{\textbf{Experiment 2}}, and the error in the measurement of the angle was much less.
