\section{Theory}

	Acoustic waves in \textbf{closed} liquids cause periodic high and low density regions in them. The spacing between high-density and low-density regions of ultrasonic waves in the MHz range is similar to the spacing in diffraction gratings. Since these density changes in liquids will cause changes in the index of refraction of the liquid, it can be shown that parallel light passed through the exciting liquid will be diffracted much as if it had passed through a grating. The experiment can be used as an indirect method to measure the velocity of sound in various liquids. The phenomenon of interaction between light and sound waves in a liquid is called the \textbf{DebyeSears effect}.

	The successive separations between two compressions or rarefactions are equal to the wavelength of the ultrasonic wave, $\lambda_u$ in the liquid. Due to reflections at the sides of the tank or the container, a stationary wave pattern is obtained with nodes and antinodes at regular intervals. In these periodic high and low density regions, the liquid acts as a diffraction grating.

	If $\lambda_u$ denotes the wavelength of sound in the liquid, $\lambda$ the wavelength of incident light in air and $\theta_n$ is the angle of diffraction of $n^{th}$ order, then we have:

	\vspace{-4mm}
	$$d\sin\theta_n = n\lambda$$

	Here, the liquid acts as the diffraction grating. So, here, the $d (= (e+b) = \frac{1}{N})$ becomes equal to $\lambda_u$. Thus:

	\vspace{-4mm}
	$$\lambda_u\sin\theta_n = n\lambda$$

	Now, if $\nu$ is the frequency of the crystal, the velocity $V_u$ of the ultrasonic wave in the liquid will be:

	\vspace{-4mm}
	$$V_u = \nu\lambda_u$$

	Thus, the velocity of the ultrasonic wave in the liquid can be calculated from the diffraction pattern. The velocity of sound in the liquid can be calculated from the velocity of the ultrasonic wave in the liquid. The velocity of sound in the liquid is given by:

	% \vspace{-4mm}

	\begin{equation}
		V_u = \frac{n\lambda\nu}{\sin\theta_n}
	\end{equation}\label{eqn:1}


	\subsection{Compressibility of Liquid (K)}

	The speed of sound depends on both an inertial property of the medium (to store kinetic energy) and an elastic property (to store potential energy).

	$$V_u = \sqrt{\frac{elastic\;property}{inertial\;property}}$$

	For a liquid medium, the bulk modulus E accounts for the extent to which an element from the medium changes in volume when a pressure is applied.The bulk modulus is a measure of the compressibility of the liquid. The bulk modulus of a liquid is given by:

	$$E = -\frac{\Delta p}{\sfrac{\Delta V}{V}}$$

	Here, $\Delta p$ is the change in pressure and $\Delta V/V$ is the fractional change in volume. The sign of $\Delta p$ and $\Delta V/V$ is opposite. So, the bulk modulus is negative. The unit of $E$ is Pascal (N/m$^2$) therefore the speed of sound in a liquid is given by:

	$$V_u = \frac{n\lambda\nu}{\sin\theta} = \sqrt{\frac{E}{\rho}}$$

	\begin{equation}
		\therefore E = V_u^2\rho = \frac{1}{K}
	\end{equation}\label{eqn:2}

	\noindent Here, $K$ = compressibility of the liquid,

	\hspace{4mm} $\rho$ = density of the liquid,

	\hspace{4mm} $E$ = bulk modulus of the liquid
