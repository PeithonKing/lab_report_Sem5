\section{Procedure}

\begin{enumerate}
	\item Follow the support manual for the spectrometer provided in the appendix for basic adjustment of the spectrometer. Determine the vernier constant of the spectrometer.
	\item Now remove the prism from the turntable. The next step is to adjust the grating on the turntable so that its lines are vertical, i.e. parallel to the axis of rotation or the turntable. Moreover, the light from the collimator should fall normally on the grating. To achieve this, the telescope is brought directly in line with the collimator so that the centre of the direct image of the slit falls on the intersection of the cross-wires (without mounting the grating). In this setting of the telescope, its vernier reading is noted; let it be $\phi$.
	\item The telescope is now turned through $90^\circ$ from this position in either direction so that the reading of the vernier becomes $(\phi+90^\circ)$ or $(\phi-90^\circ)$. Now the axis of telescope is at right angles to the direction of rays of light emerging from the collimator. The telescope is clamped in this position.
	\item The  grating of known grating element is then mounted on the grating holder, which is fixed on the turntable in such a way that the ruled surface of the grating is perpendicular to the line joining two of the leveling screws (say $Q$ and $R$) on the turntable.
	\item The table is now rotated in the proper direction till the reflected image of the slit from the grating surface coincides with the intersection of the cross-wires of the telescope.
	\item By the help of two leveling screws (Q and R), perpendicular to which grating is fixed on the table, the image is adjusted to be symmetrical on the horizontal cross-wires. The plane of the grating, in this setting, makes an angle of $45^\circ$ with the incident rays as well as with the telescope axis.
	\item The reading of vernier is now taken and with its help, the turntable is rotated through $45^\circ$ from this position so that the ruled surface becomes exactly normal to the incident rays. The turntable is now firmly clamped.
	\item The final adjustment is to set the lines of the grating exactly parallel to the axis of rotation of the telescope. The telescope is rotated and adjusted to view the first order diffraction pattern. The third leveling screw (P) of the prism table is now worked to get the fringes (spectral lines) symmetrically positioned with respect to the horizontal cross-wire.
	\item If this adjustment is perfect, the centers of all the spectral lines on either side of the direct one will be found to lie on the intersection of the cross-wires as the telescope is turned to view them one after another. The rulings on the grating are now parallel to the axis rotation of the telescope. The grating spectrometer is now fully ready to make the measurements. Do not disturb any of the setting of the spectrometer henceforth throughout the experiment.
	\item  Look through the telescope to notice the first or second order (whichever you see is completely resolved) D lines of sodium. That means you will see two yellow lines on both sides of the direct image (which is a single line) of the slit at the center. Note down the positions of the cross wire for each line on one side using the two verniers on the spectrometer. Use a torch, if needed, to read the verniers. Repeat the above step by turning the telescope to the other side too. Determine the diffraction angle, $\alpha$, for all the two spectral lines.
	\item Take two sets of reading for each D-line and calculate the corresponding wavelength $\lambda_1$ and $\lambda_2$ using \hyperref[eqn:1]{Eq. 1}. 
\end{enumerate}