\section{Conclusion}
	We see that our values of the corresponding wavelength do not agree with the literature values, indicating significant amount of error during the measurement.

	One reason of this error might be due to the rotation axis of the spectrometer table. It might not have been levelled properly with schuster's method.

	The screws of the instruments have quite high backlash error, which might also have contributed to the error.

	Also, we see that in this situation, the Fraunhoffer approximation for diffraction is quite good, and yeilds results that we expect.

	In terms of observation, I think we can improve the experiment by using digital meters for measuring the angles in spectrometer, that shall decrease much of the random error.

	Also, the readings tend to be erroneous over time of the experiment, because focusing on the small divisions of vernier scale in a dark room for long time, does affect the vision for short periods.

	The Value of $\lambda_1$ came out to be quite accurate, but the value of $\lambda_2$ is not that good (according to literature values obtained from \cite{manual}). This is because I had to use a diffraction grating of 15000 per inch, which is quite large a value. This resulted in the 2nd order angle being quite large and thus the error is large there.